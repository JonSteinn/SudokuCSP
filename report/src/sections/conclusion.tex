The project serves as a valuable lesson in improvements on backtracking.  The overhead of the conflict set in the conflict-directed backjumping is minuscule and the algorithm outperforms the other two on all nontrivial puzzles. The same can be said for AC-3 which always seems beneficial to apply. These two as a pair form the most powerful tool we built in the assignment.

Still there were some Sudoku puzzles we struggled to solve (within a reasonable time) with what we implemented. Some improvements could involve heuristics on assignment- and expansion order. An example of a heuristic for assignments could be exploring values with fewer total assignments first. An example of a heuristic for variable order could involve the size of the domain or the number of constraints to unassigned variables or even a weighted combination of both.

There are also some algorithms we did not implement such as backmarking or dynamic backtracking as well as other preproccessing methods that could help solve those puzzles our implementations are not capable of solving within a reasonable time. There are even algorithms outside the realm of CSPs which have been used to solve Sudoku such as stochastic optimization algorithms as is discussed in \cite{bib:art2}.